\documentclass[a4paper]{article}

\usepackage{fullpage}
\usepackage{caption}
\usepackage{subcaption}
\usepackage{amsmath}
\usepackage{amsthm}
\usepackage{amssymb}
\usepackage[
n,
operators,
advantage,
sets,
adversary,
landau,
probability,
notions,
logic,
ff,
mm,
primitives,
events,
complexity,
asymptotics,
keys]{cryptocode}
\usepackage{mdframed}
\usepackage[colorlinks=true, urlcolor=blue]{hyperref}

% ==== Environments ====
\newcounter{solution}
\newcounter{subsolution}[solution]
\newcommand{\solution}{\section*{Solution \stepcounter{solution}\arabic{solution}}}
\newcommand{\subsolution}{\paragraph{(\protect\stepcounter{subsolution}\alph{subsolution})}}
\newtheorem{claim}{Claim}[solution]

% ==== Custom Math Commands ====
% A few macros to get you started.
\newcommand{\keyspace}{\ensuremath{\mathcal{K}}}
\newcommand{\mssgspace}{\ensuremath{\mathcal{M}}}
\newcommand{\ctxspace}{\ensuremath{\mathcal{C}}}

\begin{document}

\section*{Collaborators}
List your collaborators here.

\solution{}
Use the \verb|\solution| command to indicate the beginning of a solution to a new problem.

\begin{claim}
    Use claims to break long proofs into smaller, manageable components. They are well suited for short, clearly defined results and improve the overall flow of the solution.
\end{claim}
\begin{proof}
    As the name suggests, the proof environment is used for writing proofs for claims (and other theorem environments).
\end{proof}

\setcounter{solution}{3}
\solution{}
The solution numbering can be changed by using the \verb|\setcounter{solution}{<value>}| command.

\solution{}
\subsolution{} The \verb|\subsolution| command can be used for writing solutions to sub-problems.

\solution{}
Use the \href{http://mirrors.ibiblio.org/CTAN/macros/latex/contrib/cryptocode/cryptocode.pdf}{cryptocode} package for other notations and drawing reduction diagrams.

\begin{solution}
    Use the \href{http://mirror.iopb.res.in/tex-archive/macros/latex/contrib/mdframed/mdframed.pdf}{mdframed} package for describing protocols and algorithms.

    \begin{mdframed}[userdefinedwidth=0.7\textwidth, align=center, frametitle=One-Time Encryption Game, frametitlerulewidth=1pt]
        \begin{enumerate}
            \item The adversary $\adv$ outputs a pair of messages $m_0, m_1 \in \mssgspace$.
            \item A key $k \leftarrow \kgen$ is generated and a uniform bit $b \in \bin$ is chosen.
                Ciphertext $c \leftarrow \enc_k(m_b)$ is computed and given to $\adv$.
            \item $\adv$ outputs a bit $b'$.
            \item The output of the experiment is 1 ($\adv$ wins) if $b' = b$ and 0 otherwise.
        \end{enumerate}
    \end{mdframed}
\end{solution}

\end{document}
